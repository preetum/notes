\documentclass[]{article}
\usepackage[T1]{fontenc}
\usepackage{lmodern}
\usepackage{amssymb,amsmath}
\usepackage{fixltx2e} % provides \textsubscript

% preetum
\usepackage{multicol}
\usepackage{wrapfig}
\usepackage{fullpage} % small margins
\usepackage[small,compact]{titlesec} % compact section titles

\usepackage{enumitem}
\setitemize{itemsep=1pt,topsep=0pt,parsep=0pt,partopsep=0pt} % compact lists
\setlist[description]{style=nextline} % desc on next line

% compact main title. see
% https://mirror.hmc.edu/ctan/macros/latex/contrib/titling/titling.pdf
% for defaults
\usepackage{titling}
\setlength{\droptitle}{-7em}
\posttitle{\par\end{center}\vspace{-2em}}
\postauthor{\end{tabular}\par\end{center}\vspace{-2.5em}}
\postdate{\par\end{center}}


% use microtype if available
\IfFileExists{microtype.sty}{\usepackage{microtype}}{}
\usepackage[unicode=true]{hyperref}
\hypersetup{breaklinks=true,
            bookmarks=true,
            pdfauthor={},
            pdftitle={},
            colorlinks=true,
            citecolor=blue,
            urlcolor=blue,
            linkcolor=magenta,
            pdfborder={0 0 0}}
\setlength{\parindent}{0pt}
\setlength{\parskip}{6pt plus 2pt minus 1pt}
\setlength{\emergencystretch}{3em}  % prevent overfull lines
\setcounter{secnumdepth}{0}

\newcommand{\normin}{\trianglelefteq}
\newcommand{\x}{\times}
\newcommand{\Z}{\mathbb{Z}}

\title{Counterexamples in (Introductory) Algebra}
\author{}
\date{\today}

\begin{document}
\maketitle

\begin{description}
\item[Isomorphism of quotients does not imply isomorphism of quotient groups \\
ie: $H \cong K \not\implies G/H \cong G/K$]
Let $G =\Z_4 \x \Z_2$, with $H = <(\bar 2, \bar 0)>$ and $K = (\bar 0, \bar 1)$. \\
Then $H \cong K \cong Z_2$ but $G / K \cong Z_4 \not\cong Z_2 \x Z_2 \cong G / H$

\item[Isomorphism of quotient groups does not imply isomorphism of quotients \\
ie: $G/H \cong G/K \not\implies H \cong K$]
(D\&F 3.3.8): For prime $p$, let $G$ be the group of $p$-power roots of unity.
And $\phi: G \to G$ be the surjective homomorphism $z \mapsto z^p$. Then $G /
\text{ker} \phi \cong G$. \\
So let $K = \text{ker}\phi$ and $H$ be trivial. Then $G/K \cong G \cong G/H$,
but $H \not\cong K$ (because $\text{ker} \phi$ is non-trivial).

\item[A group can be isomorphic to a proper quotient of itself]
Same example as above.

\item[An infinite group in which every element has finite order but for each positive
integer $n$ there is an element of order $n$]
$\prod_{n \in \mathbb{N}} Z_n$

\item[A group such that every finite group is isomorphic to some subgroup]
1) The direct product of all finite groups, or 2) The group of all bijections
$\mathbb{N} \to \mathbb{N}$ (then applying Cayley's Theorem)

\item[A nontrivial group $G$ s.t. $G \cong G \x G$]
$G = Z_2 \x Z_2 \x \cdots$, with isomorphism $(g_1, g_2, g_3, \hdots) \mapsto
((g_1, g_3, g_5, \hdots), (g_2, g_4, g_6, \hdots))$

\item[A group of order $n$ may not have a subgroup of order $k$ for all $k | n$]
The alternating group $A_4$ has order 12, but no element of order 6 (all elements are order 1, 2, or 4).

\item[Direct product of Hamiltonian Groups\footnote{non-abelian group where
every subgroup is normal} may not be Hamiltonian]
In $Q_8 \x Q_8$, the subgroup $<(i, j)>$ is not normal because $<(i, j)> = \{(1,
1), (i, j), (-1, -1), (-i, -j) \}$ but $(j, 1)(i, j)(j, 1)^{-1} = (-i, j)
\not\in <(i, j)>$

\item[Subgroups of finitely-generated groups may not be finitely generated]
The commutator subgroup of the free group on two elements $F(\{x, y\})$ cannot
be finitely generated (proof omitted).

\end{description}

\end{document}
