\documentclass{article}

\usepackage{amsmath}
\usepackage{enumerate}
\usepackage{amssymb}

\title{Math H113 - Spring 2014 Notes}

\begin{document}
\maketitle

\section*{Introduction}

This is a sparse collection of facts taken from Dummit and Foote, 3e. The
goal is to recap the most important things Prof. Vojta has covered, with an
emphasis on non-obvious results.

\section*{Confusing topics}

\begin{enumerate}[1.]
    \item 4.1 (Cycle Decompositions): The proof is hard to follow, and some
        statements (e.g `Since $G$ is a cyclic group, $G_x \trianglelefteq
        G$') are non-intuitive.
\end{enumerate}

\section*{Chapter 0/1}

\begin{enumerate}[1.]
    \item gcd: $(m, n) = am + bn$. 
    \item The Euler function: $\varphi(p^a) = p^{a-1}(p - 1)$, $\varphi(ab)
        = \varphi(a)\varphi(b)$ if $(a, b) = 1$.
    \item The dihedral group: $D_{2n} = \langle r, s \mid r^n = s^2 = 1, rs
        = sr^{-1} \rangle$.
    \item Fields ($F$, $F^{\times} = F - \{0\}$): $(F, +)$ and $(F^{\times},
        \times)$ are abelian groups. Note: if $|F| < \infty$, $\exists p,m$
        s.t $|F| = p^m$.
    \item Symmetries: $|S_n| = n!$. Disjoint cycles commute, but $S_{n
        \geq 3}$ is non-abelian.
    \item Group actions on $A$ by $G$: (i) $g_1(g_2a) = (g_1g_2)a$, (ii) $1a
        = a$. $\forall g_1,g_2,a$.
        \begin{enumerate}[(a)]
            \item Fixing $g \in G$ in the action gives $\sigma_g \in S_A$.
            \item $g \mapsto \sigma_g$ is a homomorphism ($G \rightarrow
                S_A$, the permutation representation).
        \end{enumerate}
\end{enumerate}

\section*{Chapter 2}

\begin{enumerate}[1.]
    \item Subgroup criterion: $H \not= \phi$ and $xy^{-1} \in H$ $\forall x,y
        \in H$.
    \item Centralizer: $\leq G$, commutes with $A$. Center: $\leq G$,
        commutes with $G$ itself. 
    \item Normalizer: $\leq G$ s.t $gAg^{-1} = A$ $\forall g$.
    \item Stabilizer: fixing $a \in A$, $\leq G$ s.t $ga = a$. Kernel: 
        $\forall a \in A$, $\leq G$ s.t $ga = a$.
    \item If $x^m = 1$ and $x^n = 1$, $x^{(m, n)} = 1$ (in cyclic groups).
    \item Let $x \in G$, $a \not= 0$: if $|x| = \infty$ then $|x^a| =
        \infty$. Else, if $|x| = n$, then $|x^a| = n/(n, a)$ ($\star$).
    \item Every subgroup of a cyclic group is cyclic, and cyclic groups of
        the same order are isomorphic to each other.
    \item Let $|x| = n$, $H = \langle x \rangle$. Only if $(n, a) = 1$, $H =
        \langle x^a \rangle$ (count these with $\varphi(n)$). 
        A general statement: $\langle x^m \rangle = \langle x^{(n, m)}
        \rangle$.
    \item Let $A \not= \phi$ be a set of subgroups of $G$. Then their
        intersection $\langle A \rangle = \cap A \leq G$.
\end{enumerate}

\section*{Chapter 3}

\begin{enumerate}[1.]
    \item Given $\varphi : G \rightarrow H$; $\varphi(1_G) = 1_H$, 
        $\ker \varphi \leq G$, and im($\varphi$) $\varphi \leq H$.
    \item $G/K$ is basically arithmetic on the fibers of $\varphi$, which
        are all cosets of $\ker \varphi$.
    \item The set of left cosets of any $N \leq G$ partitions $G$. However,
        the operation $uN \cdot vN = (uv)N$ is only well defined if $N
        \trianglelefteq G$ (or equivalently $N_G(N) = G$, $gN = Ng$ $\forall
        g$, or $gNg^{-1} \subseteq N$ $\forall g$).
        \footnote{A useful theorem for later: $gH = H$ iff $g \in H$.}
    \item If $|G| < \infty$ and $H \leq G$, then $|H| \mid |G|$ and 
        $|G : H| = |G|/|H|$ ($\star$).
    \item If $|G| < \infty$ and $p \mid |G|$, $\exists x \in G$ s.t $|x| =
        p$.
    \item If $|G| = p^{\alpha}m$ ($p \not| m$), $\exists H \leq G$ s.t $|H|
        = p^{\alpha}$.
    \item $\ker \varphi \trianglelefteq G$, $G/\ker \varphi \cong
        \varphi(G)$, $\varphi$ is 1-1 iff $\ker \varphi = 1$, and 
        $|G : \ker \varphi| = |\varphi(G)|$.
    \item If finite $H, K \leq G$, then $|HK| = \frac{|H||K|}{|H \cap K|}$. 
        $HK \leq G$ only if $KH \leq G$.
    \item Let $A,B \leq G$ and $A \leq N_G(B)$. Then $AB \leq G$, $B
        \trianglelefteq AB$, $A \cap B \trianglelefteq A$, and $AB/B \cong
        A/(A \cap B)$.
    \item Let $H \leq K$ and $H,K \trianglelefteq G$: then $K/H
        \trianglelefteq G/H$ so $(G/H)/(K/H) \cong G/K$.
    \item To show that a homomorphism from $\varphi : G/N \rightarrow H$ is
        well-defined, one must prove $N \leq \ker \Phi$ (with $\Phi : G
        \rightarrow H$).
    \item $(a_1 a_2 ... a_m) = (a_1 a_m)(a_1 a_{m-1})...(a_1 a_2)$. 
        The sign of a permutation (i.e the parity of the number of
        2-cycles $\epsilon(\sigma) \in \{\pm 1\}$
        \footnote{An $m$-cycle is composed of $m-1$ transpositions, 
            immediately giving $\epsilon(\sigma) = Parity(m-1)$.}
        ) is representation-independent.
    \item $\epsilon : S_n \rightarrow \{\pm 1\}$ is a surjective
        homomorphism. $\ker \epsilon = A_n$, the group of even permutations.
        Note $S_n/A_n \cong \epsilon(S_n) = \{\pm 1\}$ and $|A_n| =
        \frac{n!}{2}$.
\end{enumerate}

\section*{Chapter 4}

\begin{enumerate}[1.]
    \item $\sigma_g : A \rightarrow A$ ($a \mapsto ga$), and
        $\varphi : G \rightarrow S_A$ ($g \mapsto \sigma_g$). 
        Note: the kernel of the action $\cap_{a \in A} G_a = \ker \varphi$.
        \footnote{`Faithful' actions have kernels equal to $\{1_G\}$}
    \item For $A \not= \phi$, the actions of $G$ on $A$ and the
        homomorphisms $G \rightarrow S_A$ are bijective. Let $a \sim b$ iff $a
        = gb$ for some $g \in G$: then $\sim$ partitions $G$, and the order
        of the equivalence class (i.e orbit) containing $a$ is $|G : G_a|$.
        \footnote{`Transitive' actions induce only one orbit in $A$.}
    \item Elements in $G$ effect the same permutation on $A$ iff they're in
        the same coset of the kernel of the action.
    \item Let $H \leq G$, $A$ be the set of left cosets of $H$ in $G$, and
        $G$ act on $A$ (with $\pi_H : G \rightarrow S_A$). Then the action
        is transitive, $G_{1H} = H$, and $\ker \pi_H = \cap_{x \in G}
        xHx^{-1}$ (giving the largest normal subgroup of $G$ in $H$).
    \item If $|G| = n$, $G \cong H$ for some $H \leq S_n$. If $p$ is the
        smallest prime s.t $p | n$, then any subgroup $H \leq G$ s.t $|G :
        H| = p$ is normal.
\end{enumerate}

\section*{Chapter 5}

\begin{enumerate}[1.]
    \item Given a direct product $G_1 \times G_2 \times ... \times G_n$,
        $G_i \cong \{(1, ..., g_i, ..., 1) \mid g_i \in G_i\}$.
        \footnote{The projection $\pi : G \rightarrow G_i$ is $g \mapsto
        g[i]$.}
    \item Let $G = \langle A \rangle$ ($A \subseteq G$, finite). Then $G \cong
        \mathbb{Z}^r \times Z_{n_1} \times Z_{n_2} \times ... \times
        Z_{n_s}$ s.t $r \geq 0$, $n_j \geq 2$ $\forall j$, and $n_{i+1} |
        n_i$ for $1 \leq i < s$ uniquely (up to isomorphism).
    \item Let $n = \Pi n_i$: if $p | n$ then $p | n_1$. If $n$ is a product
        of distinct primes, then $Z_n$ is the only abelian group of order
        $n$ (up to isomorphism).
    \item Let $n = p_1^{\alpha_1}...p_k^{\alpha_k}$. Then $G \cong A_1
        \times ... \times A_k$ where $|A_i| = p_i^{\alpha_i}$. Each $A_i
        \cong Z_{p_i^{\beta_1}} \times ... \times Z_{p_i^{\beta_t}}$ where
        $\beta_i \geq \beta_{i+1}$ and $\sum_i^t \beta_i = \alpha_i$.
    \item $Z_m \times Z_n \cong Z_{mn}$ iff $(m, n) = 1$, so $Z_n \cong
        Z_{p_1^{\alpha_1}} \times ... \times Z_{p_k^{\alpha_k}}$.
    \item The group exponent is the smallest positive integer s.t $x^n = 1$
        $\forall x \in G$.
\end{enumerate}

\section*{Chapter 6}

\begin{enumerate}[1.]
    \item Let $F(S)$ be the group of words formed from $S$.
        Given a map $\psi : S \rightarrow G$, there exists a unique
        homomorphism $\Phi : F(S) \rightarrow G$ s.t $\Phi|_S = \psi$.
    \item Because $\Phi$ is a homomorphism, $\Phi(s_i^{\epsilon_i} ...) =
        \psi(s_i)^{\epsilon_i} ...$.
    \item Empty word: $(1, 1, ...)$. Reduced word: $s_{i+1} \not= s_i^{-1}$
        and $s_i = 1 \Rightarrow s_{k \geq i} = 1$.
    \item Subgroups of free groups are also free groups.
    \item Let $S \subseteq G$ s.t $G = \langle S \rangle$. A presentation of
        $G$ is some $(S, R)$ s.t $\ker \Phi$ is the smallest normal subgroup
        containing $\langle R \rangle$. $G$ is finitely generated if $S$ is
        finite, and finitely presented if $R$ is also finite.
\end{enumerate}

\section*{Chapter 7}

\begin{enumerate}[1.]
    \item Rings: $(R, +)$ is abelian, $\times$ is associative/distributive.
        If $\times$ commutes, so does $R$. If $1 \in R$, $1r = r1$ $\forall
        r$. If $1 \not= 0$ and every nonzero element has an inverse, $R$ is
        a division ring. Commutative division rings are fields.
    \item Let $a,b \in R$ be nonzero. Then $a0 = 0a = 0$. Zero divisors: 
        $ab = 0$ or $ba = 0$. The set of units (i.e $uv = vu = 1$) is
        $R^{\times}$.
    \item Integral domains are commutative rings (with $1 \not= 0$)
        with no zero divisors. Zero divisors cannot be units, therefore
        fields have no zero divisors. 
    \item Finite integral domains are fields. Subrings are $\leq R$ and
        closed under $\times$.
    \item $R[x]$: $(ab)x^k = \sum_{i=0}^k a_ib_{k-i} =
        \sum_{i=0}^k a_{k-i}b_i$. Note: $R \subset R[x]$ (as the constant
        polynomials) and $R[x]$ is commutative by definition.
    \item If $R$ is an integral domain, $\deg(ab) = \deg(a) + \deg(b)$,
        $R[x]^{\times} = R^{\times}$, and $R[x]$ is an integral domain.
    \item Square matrices: $(a_{ij}) \in M_n(R)$. Invertible: $GL_n(R)$.
    \item Fix a commutative ring $R$ with $1 \not= 0$ and let
        $G$ be a finite group. Group rings $RG$ contain all formal sums
        $\sum_i r_ig_i$ $r_i \in R$. Addition is done componentwise, and
        $RG$ always has zero divisors.
    \item Ring homomorphism: $\varphi : R \rightarrow S$ s.t $\varphi(a + b) =
        \varphi(a) + \varphi(b)$ and $\varphi(ab) = \varphi(a)\varphi(b)$.
        $\ker \varphi = \{r \in R \mid \varphi(r) = 0_S\}$, as if $\varphi$
        were a group homomorphism.
    \item $I = \ker \varphi$ is a subring of $R$ (and an ideal/normal subgroup
        thereof), im($\varphi$) is a subring of $S$. 
        If $\alpha \in \ker \varphi$, then $r\alpha$, $\alpha r \in R$
        $\forall r \in R$.
    \item Ideals: if $rI \subseteq I$, $Ir \subseteq I$, and $I$ subring of
        $R$. $R/I$ is a quotient ring s.t $(r+I)+(s+I) = (r+s)+I$ and
        $(r+I)\times(s+I) = rs + I$. $R/\ker \varphi \cong \varphi(R)$.
        Note: every ideal is the kernel of a ring homomorphism and vice
        versa.
    \item Let $A$ be a subring and $B$ an ideal of $R$. Then $A + B$ is a
        subring of $R$, $A \cap B$ is an ideal of $A$, and $(A+B)/B \cong
        A/(A \cap B)$.
    \item Let $I \subseteq J$ be ideals of $R$, then $(R/I)/(J/I) \cong
        R/J$.
    \item Ideal math: $I+J = \{i+j \mid i \in I, j \in J\}$, $IJ$ is the set
        of all finite sums of elements of the form $ij$, and $I^n$ are all
        $n$-length products within $I$.
\end{enumerate}

\end{document}
